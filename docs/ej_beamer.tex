\documentclass{beamer}
\usepackage[spanish]{babel}
\usepackage[utf8]{inputenc}
\usepackage{graphicx}


\begin{document}



%+++++++++++++++++++++++++++++++++++++++++++++++++++++++++++++++++++++++
\begin{frame}
\title[Presentación]{Imforme de número $\pi$}
\author{Ana Gómez Pérez\\ Práctica de Laboratorio 10}
\date{11 de abril de 2014}


\begin{abstract}
El objetivo de este documento es saber más sobre el número $\pi$ y exponerlo en un pdf con beamer\alert{Matemáticas}~\cite{guia} en la universidad (figura)~\ref{et:graf} .
\end{abstract}
\end{frame}

%++++++++++++++++++++++++++++++++++++++++++++++++++++++++++++++++++++++
\begin{frame}
\section{historia del número $\pi$}
$\pi$ es un número irracional, cociente entre la longitud de la circunferencia y la longitud
de su diámetro. Se emplea frecuentemente en matemáticas, física e ingeriería. El valor numérico
de $\pi$ truncado a sus diez primeras posiciones decimales, es el siguiente: 3,1415926535...o
Dividiend
\end{frame}

%++++++++++++++++++++++++++++++++++++++++++++++++++++++++++++++++++++++
\begin{frame}
\section{Historia del valor $\pi$}
La búsqueda del mayor número de decimales de número $\pi$ ha supuesto un esfuerzo constante de numerosos científicos a lo largo de la historia.
Algunas aproximaciones históricas de $\pi$ son las siguientes:
 \subsection{Antiguo Egipcio}
 \[S=\pi r^2 \simeq (\frac{8}{9}d)^2 = \frac{64}{81}d^2 = \frac{64}{81}(4r^2)\]
 \subsection{Antigua Mesopotánia}
 \[ \pi \approx 3+\frac{1}{8}=3,125\]


\end{frame}

\begin{frame}
  \frametitle{Bibliografía}

  \begin{thebibliography}{10}
   
  \end{thebibliography}

    \beamertemplatebookbibitems
    \bibitem[Matemáticas]{guia}  
    Guía Beamer. 
    {\small $http://www4.ujaen.es/~jmmoreno/latex/clatex2.pdf$}


   \end{thebibliography}
\end{frame}
%+++++++++++++++++++++++++++++++++++++
\end{document}